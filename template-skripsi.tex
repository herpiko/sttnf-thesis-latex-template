%-------------------------------------------------------------------------------
%                    Template Naskah Skripsi STT-NF
%                 Di-forking dari format JTETI FT UGM
% 						(c) @gunturdputra 2014
%-------------------------------------------------------------------------------

%Template pembuatan naskah skripsi.
\documentclass{sttnfskripsi}

%Untuk prefiks pada daftar gambar dan tabel
\usepackage[titles]{tocloft}
\renewcommand\cftfigpresnum{Gambar\  }
\renewcommand\cfttabpresnum{Tabel\   }

%Untuk hyperlink dan table of content
\usepackage{hyperref}
\newlength{\mylenf}
\settowidth{\mylenf}{\cftfigpresnum}
\setlength{\cftfignumwidth}{\dimexpr\mylenf+2em}
\setlength{\cfttabnumwidth}{\dimexpr\mylenf+2em}

%Untuk Bold Face pada Keterangan Gambar
\usepackage[labelfont=bf]{caption}

%Untuk caption dan subcaption
\usepackage{caption}
\usepackage{subcaption}


%-----------------------------------------------------------------
%Di sini awal masukan untuk data proposal skripsi
%-----------------------------------------------------------------
\titleind{Templat Skripsi STT Nurul Fikri dengan Latex}

\fullname{Herpiko Dwi Aguno}

\idnum{0110215036}

\approvaldate{5 Oktober 2017}

\degree{Sarjana Komputer}

\yearsubmit{2017}

\program{Teknik Informatika}

\dept{Teknik Informatika}

\firstsupervisor{Hilmy Abidzar Tawakal, ST., M.Kom.}
\firstnip{0411118402}

\secondsupervisor{Second Supervisor}
\secondnip{0000000}

\firstenemy{Dosen Penguji I}
\firstenemynip{0000000}

\secondenemy{Dosen Penguji II}
\secondenemynip{0000000}


%-----------------------------------------------------------------
%Disini akhir masukan untuk data proposal skripsi
%-----------------------------------------------------------------

\begin{document}

\cover

\secondcover

\approvalpage

\noplagiatstatement

%-----------------------------------------------------------------
%Disini awal masukan Acknowledment
%-----------------------------------------------------------------
\acknowledgment
\begin{flushright}
\emph{Teruntuk yang tercinta,\\ Komisi Pemilihan Umum Republik Indonesia.}
\end{flushright}

%-----------------------------------------------------------------
%Disini awal masukan untuk Prakata
%-----------------------------------------------------------------
\preface
Assalamu'alaikum Wr. Wb.

\vspace{0.5cm}

Puji syukur penulis panjatkan ke hadirat Tuhan Yang Maha Esa karena hanya dengan rahmat dan hidayah-Nya, skripsi ini dapat terselesaikan tanpa halangan berarti. Keberhasilan dalam menyusun laporan penelitian ini tidak lepas dari bantuan berbagai pihak yang mana dengan tulus dan ikhlas memberikan masukan guna sempurnanya penelitian ini. Oleh karena itu dalam kesempatan ini, dengan kerendahan hati penulis mengucapkan terima kasih kepada:

\begin{enumerate}
\item{Bapak Ahmad Rio A, S.Si., M.Si, selaku Ketua Prodi Teknik Informatika STT Nurul Fikri,}
\item{Bapak Hilmy Abidzar Tawakal, ST., M.Kom sebagai pembimbing pertama dan Bapak Second Supervisor. sebagai pembimbing kedua, yang telah memberikan banyak bantuan, bimbingan, serta arahan dalam penelitian ini,}
\item{Teman-teman dari komunitas-komunitas perangkat lunak terbuka yang telah banyak berbagi kesempatan untuk berdiskusi.}
\end{enumerate}

Penulis menyadari bahwa penelitian ini masih belum sempurna. Kritik dan saran dapat ditujukan langsung pada e-mail saya di herpiko@gmail.com. Akhir kata penulis mohon maaf yang sebesar-besarnya apabila ada kekeliruan di dalam penulisan laporan penelitian ini.

\vspace{0.5cm}

Wassalamu'alaikum Wr. Wb.

\begin{tabular}{p{7.5cm}c}
&Depok, 5 Oktober 2017\\
&\\
&\\
&\textbf{Penulis}
\end{tabular}

%-----------------------------------------------------------------
%Disini akhir masukan untuk muka skripsi
%-----------------------------------------------------------------
\tableofcontents
\addcontentsline{toc}{chapter}{DAFTAR ISI}
\listoftables
\addcontentsline{toc}{chapter}{DAFTAR TABEL}
\listoffigures
\addcontentsline{toc}{chapter}{DAFTAR GAMBAR}

%-----------------------------------------------------------------
%Daftar Singkatan [Optional]
%-----------------------------------------------------------------
\singkatan
\noindent

\begin{tabular}{p{20pt}p{3pt}l}
\textbf{P}\\
    PKI & & Public Key Infrastructure\\
\\
\end{tabular}


%-----------------------------------------------------------------
%Disini awal masukan Intisari
%-----------------------------------------------------------------
\begin{abstractind}
Lorem ipsum dolor sit amet, consectetur adipiscing elit. Quisque in imperdiet nisl. In sit amet tellus in dolor interdum rhoncus eu non metus. Sed et nulla quis orci viverra lobortis. Nullam ultricies facilisis felis, sed mattis nisl accumsan pellentesque. Sed dictum lacus ipsum, sit amet imperdiet turpis finibus ut. Suspendisse potenti. Cras egestas leo ipsum, at elementum lectus pulvinar ut. Mauris non mi enim. Cras eu feugiat ex.

Donec sollicitudin pulvinar lacinia. Nulla a dolor vestibulum, fermentum orci a, ultrices velit. Mauris gravida luctus libero, at iaculis mi volutpat sed. Morbi ut ex a justo eleifend egestas sit amet non nisi. Donec porta sollicitudin metus, vel aliquet magna posuere sed. Nulla condimentum maximus iaculis. Nam urna magna, vehicula at sapien sed, elementum fermentum metus. Donec dictum posuere tortor at feugiat. Nam eu suscipit purus. Fusce pellentesque facilisis interdum. Nulla in rhoncus justo. Donec sollicitudin neque augue, id pretium metus dictum in. Vivamus porta faucibus tristique. Maecenas dictum mauris ac condimentum tempor. Ut vel ipsum tempor turpis feugiat dictum et ut libero.

\bigskip
\noindent
\textbf{Kata kunci :} \emph{public key infrastructure}, \emph{cryptography}.
\end{abstractind}

\begin{abstracteng}
\emph{
Lorem ipsum dolor sit amet, consectetur adipiscing elit. Quisque in imperdiet nisl. In sit amet tellus in dolor interdum rhoncus eu non metus. Sed et nulla quis orci viverra lobortis. Nullam ultricies facilisis felis, sed mattis nisl accumsan pellentesque. Sed dictum lacus ipsum, sit amet imperdiet turpis finibus ut. Suspendisse potenti. Cras egestas leo ipsum, at elementum lectus pulvinar ut. Mauris non mi enim. Cras eu feugiat ex.}

\emph{
Donec sollicitudin pulvinar lacinia. Nulla a dolor vestibulum, fermentum orci a, ultrices velit. Mauris gravida luctus libero, at iaculis mi volutpat sed. Morbi ut ex a justo eleifend egestas sit amet non nisi. Donec porta sollicitudin metus, vel aliquet magna posuere sed. Nulla condimentum maximus iaculis. Nam urna magna, vehicula at sapien sed, elementum fermentum metus. Donec dictum posuere tortor at feugiat. Nam eu suscipit purus. Fusce pellentesque facilisis interdum. Nulla in rhoncus justo. Donec sollicitudin neque augue, id pretium metus dictum in. Vivamus porta faucibus tristique. Maecenas dictum mauris ac condimentum tempor. Ut vel ipsum tempor turpis feugiat dictum et ut libero.
}

\bigskip
\noindent
\textbf{Keywords :} \emph{public key infrastructure}, \emph{cryptography}.
\end{abstracteng}
%-----------------------------------------------------------------
%Disini akhir masukan Intisari
%-----------------------------------------------------------------

%-----------------------------------------------------------------
%Disini awal masukan untuk Bab
%-----------------------------------------------------------------
\include{bab1}

\include{bab2}

\include{bab3}

\include{bab4}

\include{bab5}

%-----------------------------------------------------------------
%Disini akhir masukan Bab
%-----------------------------------------------------------------


%-----------------------------------------------------------------
% Disini awal masukan untuk Daftar Pustaka
% - Daftar pustaka diambil dari file .bib yang ada pada folder ini
%   juga.
% - Untuk memudahkan dalam memanajemen dan menggenerate file .bib
%   gunakan reference manager seperti Mendeley, Zotero, EndNote,
%   dll.
%-----------------------------------------------------------------
\bibliography{IEEEabrv,daftar-pustaka}
\addcontentsline{toc}{chapter}{DAFTAR PUSTAKA}
%-----------------------------------------------------------------
%Disini akhir masukan Daftar Pustaka
%-----------------------------------------------------------------

\end{document}